
\documentclass[onecollarge,runningheads]{svjour2}
% 4 July 2017
\smartqed  % flush right qed marks, e.g. at end of proof
\usepackage{graphicx}
\usepackage{amssymb}
\journalname{Journal of Biological Physics}
%\newcommand{\nexists}{\not\exists}
\begin{document}
\title{Insert your title here%\thanks{Grants or other notes
%about the article that should go on the front page should be
%placed here. General acknowledgments should be placed at the end of the article.}
}
\subtitle{Do you have a subtitle?\\ If so, write it here}
%\titlerunning{Short form of title}        % if too long for running head
\author{Jaime Gomez-Ramirez        \and
        Jose Luis Perez-Velazquez %etc.
}
%\authorrunning{Short form of author list} % if too long for running head
\institute{The Hospital for Sick Children \at
			  Department of Neuroscience and Mental Health \\	              
              Bay, 686 Toronto Canada   \\
              \email{jaime.gomez-ramirez@sickkids.ca}           %  \\
%             \emph{Present address:} of F. Author  %  if needed
           \and
           S. Author \at
              second address
}
\date{Received: date / Accepted: date}
% The correct dates will be entered by the editor
\maketitle

\begin{abstract}
The pace at which data is being produced poses important challenges at either conceptual and computational level. The high dimensionality, heterogeneity and complexity of biological data has deeply transformed the way in which life scientists work. Network theory, machine learning and stochastic modeling are now standard de facto in systems biology and neuroscience. Mathematics is however protean and new languages that deal with the curse of dimensionality problem and the network threshold selection problem are available. We introduce a new multiscale network framework based on algebraic topology and category theory able to effectively deal with the shortcomings of network theory and clustering.

\keywords{metabolic network systems biology \and Category Theory  \and Graph Theory \and Functor\and emergent properties\and Graph Theory\and Algebraic topology \and Persistent homology \and Causality}
\end{abstract}

\section{Introduction}

% The ontological primacy of relations
Following the steps of Poincar{\'e}, modern physics teaches us that what science can attain is not the knowledge of things themselves but only the relations between the things \cite{Poincare:1952}. The physical world is a network of interacting systems with reciprocal information and satisfactory understanding of how this interaction evolves is the primordial goal of science \cite{rovelli2015relative}.
The interaction can be precisely defined in mathematical terms and quantified as a correlation or distance measure between two variables.

For the sake of clarity and without loss of generality we frame the problem using statistical terminology.
For a given data set each data point represents a collection of properties, for example, spatial coordinates, weight, age, temperature, pressure, time series etc. Data points are related to each other based on a distance measure which can be defined quite liberally depending on the needs of the problem at hand. For example, the "closeness" of two data points can be expressed, among others, as a metric distance, a correlation coefficient or a coherency coefficient when the data points represent time series.
The closeness is always built upon the properties of the data points and requires a threshold. Thus, two points are close to each other when the distance among them is within a certain threshold. The choice of the threshold makes factual the relationship between data points. Two or more points are close to each other when the distance is within a certain threshold.

Network modeling has became a standard de facto to study the spatial properties of biological systems. 
Although we must be prudent in using the term paradigm shift, the wealth of studies linking network models with a range of phenomena as wide as genetic marker, genetic regulatory processes, viral evolution, neural coding, synaptic plasticity. cognition, neurodegenerative disorders and even consciousness, underlie the inescapability of network modeling.  
It is worth pointing out, however, that a graph is a set of nodes or vertices, some of them connected forming edges or links. Graphs are built by connecting two elements, that is to say, the only type of relationship possible is between pairs. The limitation of having only dyadic (or bivariate in statistics jargon) relationships have important limitations that should not be overlooked \cite{giusti2016two} .

Binary or unweighted networks are the result of thresholding or selecting the significance level. This is problematic, particularly when the underlying system is  not scale invariant. Small world or clusterness are joint measures and can change very drastically depending on the choice of the threshold \cite{toppi2012statistica}. Furthermore, by adopting a threshold, we may be loosing important information, for example, it may occur that some small-scale features are noise artifacts while other are critically important \cite{fallani2014graph}, \cite{papo2014complex}.

Algebraic topology \cite{munkres1984elements} provide a language and a methodology to overcome these limitations.
% Give some references, some context
%https://elifesciences.org/content/3/e03476   %PUT REFS Ojo esto viene casi tal cual del papaer de Giusti
The emerging field of topological biology has already produced important insights, including among others, the discovery of new genetic markers for breast cancer survival \cite{nicolau2011topology}, measurement of structure and stability of biomolecules (Gameiro et al. 2013; Xia et al. 2015), new frameworks for understanding viral evolution (Chan et al. 2013), characterization of dynamics in gene regulatory networks \cite{boczko2005structure} quantification of contagion spread in social networks (Taylor et al. 2015), characterization of structure in networks of coupled oscillators (Stolz 2014), the study of phylogenic trees (Miller et al. 2015), and the classification of dicotyledonous leaves (Katifori and Magnasco 2012). 

The main goal of this paper is to introduce this new language and explain what has to offer to mathematical biologists, biophysicists or anybody interested in modeling spatio-temporal patterns in biological systems.
We describe simplicial complexes, a generalization of graphs able to capture polyadic or multiple coactivity patterns and the Rips complex, an algebraic formalism that captures the invariance in network dynamics. %Funtors category theory?
Section \ref{sec:s1} provides an introduction to algebraic topology, specifically simplicial complexes and the rips filtration and functorality. In section \ref{sec:1} we justify the need of a multiscale framework by exposing the inherent difficulties of the existing paradigm to deal with the threshold selection problem and to encoding multiple coactivity patterns.
We conclude (section \ref{sec:1}) with a discussion on the future of algebraic topology in biological systems modeling and its implications in a causal theory of biological function.

\section{Algebraic topology for biologists}
\label{se:s1}

In this section we will provide some definitions and provide examples to make the concepts clear to the reader, specially for those not familiar with the field of topology. In particular, we provide definitions of alegraic topology, topology, topological space and simplicial complex. The idea is to give the reader the basic idea, for a more in detail treatment of algebraic topology see \cite{munkres1984elements}, \cite{hatcher2001} and for an introduction to topological data analysis see \cite{carlsson2009topology} and \cite{wasserman2016topological}. The reader familiar with these terms can directly skip to section \ref{}.

Algebraic topology is the study of topological spaces using abstract algebra. The goal is to classify topological spaces up to homeomorphism, that is, find the intrinsic qualitative aspects of spatial objects that remain invariant under homeomorphic transformations. Definitions of homeomorphism, continuous maps and related terms can be find in the Appendix.
The are two main approaches in the study of algebraic topology: one is to decompose the topological space into many but simple pieces and the other is to deal with only a few but complicated pieces. Computational topologists find more interesting the first approach -many simple pieces- and the simplicial complex is the prime example. The second approach -few but complicated pieces- is of more interest for those looking for aesthetically pleasing shapes \cite{edelsbrunner2010computational}. %algebraic geometry

Since the goal of algebraic topology is to classify topological spaces we need to know first what a topological spaces is and for that we need to define topology.

%Definition of Topology. A topology is a set of sets
Let M be a set, a topology O is a subset of the power set of M, $O \subseteq \mathcal{P}(M)$ iff O  satisfies three axioms:
\begin{itemize}
\item[i] $\emptyset \subseteq O, M \subseteq O$
\item[ii] $U \exists O, V \subseteq O$, then, $U \cap V \subseteq O$
\item[iii] $U_i \subseteq O, \bigcup  U_i \subseteq O$
\end{itemize}
The first axiom states that the empty set and the set M are in the topology O, the second axiom that for any two subsets of O the intersection also exists in the topology O, and the third axiom that the union of any of subset of O, also belongs to O. 

Example 1: Let $M = \{a,b,c\}$ and $O = \{\emptyset, \{a,b,c\}\}$. O is a topology since it holds the three axioms.
Example 2: Let $M = \{a,b,c\}$ and $O = \{\emptyset, \{a\}, \{b\}\}$. O is not a topology since axiom ii does not hold true, $\{a\} \cup \{b\} \nsubseteq O$.

For any given set M, the topologies on the set M are bounded. Thus, the topology with the fewest number of elements is $O = \{\emptyset, M\}$ (chaotic topology), and the topology with the maximum number of elements is $O = P(M)$ (discrete topology). 
%A more useful topology is the standard topology which is defined as follows. For a set $M = \mathbb{R}^d = \mathbb{R} \times \mathbb{R} \times \mathbb{R} ... \mathbb{R}$, $O \exists \mathcal{P}(\mathbb{R}^d)$. The soft ball for any number $r \exists \mathbb{R}^+$ and point $p \exists  \mathbb{R}^d$, $B_r(p) = \{q_1...q_d\}| \sum_{i=1}^{d}(q_i - p_i)^2 < r^2$.

%Example  min 24 %https://www.youtube.com/watch?v=7G4SqIboeig

%The goal of defining topologies on a set is to add structure to that set, in particular, continuity. Sets are collection points lacking structure, in order to study how the objects in a set are related data, for example who is connected with who, we need to find continuity maps. 

A topological space is a set with the structure provided by the chosen topology. Topological spaces are continuous models of reality. Of note, if we are only interested in connectedness, topological spaces are, in practical terms, similar to graphs.
Topological spaces are the most general mathematical space, for instance, metric spaces are specializations of topological spaces equipped with extra structure or constraints.  
Topological spaces allows for the study of connectivity and continuity. For example, two points are topologically close if the are in the same neighborhood and the neighborhood of a point is an open set that contains that point. 

Formally, a topological space is an ordered pair or tuple consisting in a set M (ZFC axioms for example) and a topology O, (M,O). The topology O is a  a collection of subsets of M, satisfying the axioms defined above.
The elements of the topology O are called open sets. Formally, if the subset U, $U \subseteq M$ and U is in the topology O, $U \exists O$, then U is called an open set. 
%If the complement of A, $M\A$, belongs to the topology O, $M\A \exists O$, then A  is a closed set. 
% of course (A \exists M)
In conclusion, a topology on a point set M is a collection of subsets of M, called open sets, such that M and the empty set are open and the intersection and union of subsets are open as well. For example, the set of points in the plane $M \subseteq \mathbb{R}^2$ together with the topology generated by the Euclidean metric (pairs of points) is a topological space. To construct it, we call the set of points at distance less than $r, r > 0$ from a point $x \subseteq \mathbb{R}^2$ an open disk. Taking finite intersections and arbitrary unions of open disks, we get a collection of open sets that satisfies the conditions of topological space. A neighborhood is an open disk centered in the point. 

Importantly, the notion of an open set allows us to predicate on the nearness of points in a topological space, without explicitly having a concept of distance defined, that is, without imposing, for example, an euclidean distance between the data points. Once a choice of the topology is made, directly given by the open set notion, it becomes possible to study the properties of continuity, connectedness, and compactness of the data set. 
%Three points close to each other are said to be coactive (more on this in section ).
%Each choice of open sets for a space is called a topology.

\subsection{Simplicial complex} 
%Simplicial complex is an object from algebraic topology that deals satisfactorily with this problem.
%The boundary of a simplex consists of all possible subsets of its constituent vertices, called its faces. 
%A simplicial complex encodes polyadic relations through its simplices.
%Because any given simplex is required to “contain all of its faces”, it suffices to specify only the maximal simplices, those which do not appear as faces of another simplex (Fig. 4c). This dramatically reduces the amount of data necessary to specify a simplicial complex, which helps make both conceptual work and computations feasible.
%YS: ojo literal

%(called simplex \footnote{A simplex is a subset of affinely independent points.})
Once we have defined topological space we can define simplicial complex. 
A simplicial complex is an algebraic topology object and is a generalization of graph.
A simplicial complex is a decomposition of a topological space into (topologically) simple pieces and their common intersections are lower-dimensional pieces of the same kind. 
Formally, an (abstract) simplicial complex X, is a pair of sets: $V_X$ of vertices and $S_X$ of simplices. Each simplex in the set of simplices $S_X$ is a subset of the vertices $V_X$ plus the condition that if $\sigma$ is in $S_X$ then every subset $\tau$, $\tau \in \sigma$ is also in $S_X$, $\tau \in \S_X$. 

%YS
Example 3.

A simplex with n elements (vertices) is called a $(n-1)-$simplex or what is the same the convex hull of n vertices. Simplices of small dimensions have proper names, a vertice is a 0-simplex, an edge is a 1-simplex, a triangle or face is a 2-simplex and a tetrahedron is a 3-simplex \footnote{YS: Platonic forms}. A graph is a particular case of a simplex, a vertice in a graph is a 0-simplex and an edge is a 1-simplex. In the same way as one can represent a graph as a collection of points and line segments between them, one can represent the simplices in a simplicial complex as a collection of solid regions spanning vertices.
% A graph consists on a set of vertices and a set of edges or pair of vertices. A simplicial complex, on the other hand, consists of a set of vertices and a collection of simplices where a simplex is a finite set of vertices with the additional property that any subset of a simplex is also a simplex. YS: Captioin of figure
% and subsets $\tau \in \sigma$ are faces of $\sigma$.

%Isabel K. Darcy simplicial complex from data https://www.youtube.com/watch?v=E7d7LHb6PLc&t=902s
Let us see with an example. Given a set of data points shown in \ref{fig:} where each point is, for example, a time series measured at a particular region of interest. Each point is also a 0-simplex, in order to study the structure of the data set we need to relate the points based on any idea of closeness that is relevant for our application, for example the correlation between two time series.
%How to grow simplices of higher order? 
The first step is to connect the points that are close to each other according to a threshold, for example  significant correlation level. We draw an edge or 1-simplex for every two close data points that are close to each other. As a result, we have  cast the dataset into two disjoint sets or clusters of 1-simplices. The resulting clustering is based on one dimensional simplices. But why to stop there and do not study clustering (shape) at higher dimensions
The next step is to find 2-simplices, when a triangle can be we filled in we get the "side" or a 2-simplex. We can continue finding 3 dimensional simplices or tetrahedron and so on.

Figure Simplex (1 threshold)

% YS here goes the figure with an example on how to build a measure the structure of simplicial complexes
% figure caption {With n points that are closed together, I can form a n-1 simplex. So a data point is a 0-simplex, when we build a relationship we have 1-simplex and whenever we have a triangle we add 2-simplices, whenever we have a tetrahedron (4 points close together.  tetrahedron is the Platonic solid $P_5$ with four polyhedron vertices, six polyhedron edges, and four equivalent equilateral triangular faces) we have a 3-simplex, when 5 points are lose together we add 4 -simplex and so on. It is in reality a two steps process, one we connect the points and two we build a n-complex for each group of n+1 points.}

%\subsubsection{How do we measure the structure of simplicial complexes?}
The choice of the threshold affects the kind of simplicial complex that we get, since choosing one in particular is hardly justifiable, we choose all possible thresholds. By increasing the threshold eventually all the data points will be part of the same simplex, one component.

Figure Simplicial Complex (All thresholds)

\subsubsection{Filtration of simplicial complexes} 
%Simplicial complexes encode diverse neural data modalities. After choosing the complex of interest, quantitative and theoretical tools can be used to describe, compare, and explain the statistical properties of their structure in a manner analogous to graph statistics or network diagnostics.
Just as with network models, once we have effectively encoded the data of interest in a simplicial complex, it is necessary to find useful quantitative measurements of the resulting structure to draw conclusions about the system from which the data were collected.
We start by generalizing familiar graph statistics to be used in simplicial complexes. The simplest local measure of structure – the degree of a vertex – naturally becomes a vector-measurement whose entries are the number of maximal simplices of each size in which the vertex participates, this vector is more intuitively thought of as a generalization of the clustering coefficient of the vertices. Other local and global statistics such as efficiency and path length can be generalized by considering paths through simplices of some fixed size, which provides a notion of robust connectivity between vertices of the system (Dlotko et al. 2016).
In addition to these more or less straight extensions of quantitative graph measure into simplicial complexes, however the power of algebraic topology comes with the use of algebraic topology for example, homology of the complex to identify homology cycles.

% filtration to binarize rather than thresholding
The filtration of the complex is the method by which the complex is reduced into its constituent pieces. Filtrations can be used to assess the dynamics of neural processes. Filtrations reveal more detailed structure in the complex, and provide tools for understanding how that structure arises and can be seen as an alternative to thresholding a weighted complex, providing a principled approach to binarizing graphs.
A filtration of complexes can be constructed by consecutively applying each of the weights as thresholds in turn, constructing an unweighted simplicial complex whose simplices are precisely those whose weight exceeds the threshold, and labeling each such complex by the weight at which it was binarized. However, it is also the case that these unweighted complexes are related to one another, and more sophisticated measurements of structure, like homology, can exploit these relations to extract much finer detail of the evolution of the complexes as the threshold varies.

YS: Example of filtration 

With optogenetics, DBS, TMS etc. one can stimulate single neurons or specific groups of neurons to control their function. To meet this need, one can construct a different type of filtration \cite{taylor2015topological} that construct a sequence of simplicial complexes with a time parameter, labeling each simplex as “on” or “off” at each time, and require that once simplices “turn on” they remain so indefinitely. This type of filtration is of use in  diffusion model of fronto-temporal dementia and in contagion models (Taylor et al. 2015), where a simplex becomes active once sufficiently many nearby simplices are active.

\subsection{Persistent homology}
\label{sec:1}
Persistent homology:  the sequence of complexes in the filtration is transformed by homology into an inter-related family of evolving cycles.

Computational problems: computing the persistent homology from correlation data of 100 neurons leads to a simplicial complex with approximately 107 4-simplices, while the same computation for a population of 200 neurons involves two orders of magnitude more.
To mitigate this combinatorial growth in complexity, use reduction algorithms as a sort of preprocessing to collapse the size of the complex.
Another approach is to use the fact that homology can be computed locally and then aggregated, allowing for distributed computation over multiple processors and memory cores (Bauer et al. 2014). Finally, computing approximate homology further reduces complexity in difficult cases while still providing useful statistical information (Sheehy 2013). 
For a review in the state of the art of computational methods in topological analysis (persistent homology) \cite{otter2015roadmap}.

Persistent homology (PH) is a method used in topological data analysis (TDA) to study qualitative features of data that persist across multiple scales. It is robust to perturbations of input data, independent of dimensions and coordinates, and provides a compact representation of the qualitative features of the input. (Python Mapper software) 
% www.ams.org/journals/bull/2009-46-02/S0273-0979-09-01249-X/ http://diglib.eg.org/handle/10.2312/SPBG.SPBG07.091-100
% $MapperGUI.py


A homeomorphism between two topological spaces X,Y is a bijective function such that $f:X \to Y$ and $f^{-1}$ are both continuous. If topological spaces are homeomorphic then they are topologically equivalent, that is, have the same topology type.

The basic question in topology is to classify spaces up to topology type. 
Note that the problem of classification is ill-defined it depends on what we have in mind we want and how to classify, classification or clustering is subjective science, despite these problems we still need to classify because this is the only way we can understand reality. %min 7. https://vimeo.com/105644331

Given a topological space, its homology is a formal, algebraic way to talk about its connectivity.
%NETWORK COMPARISON WITH FREQUENCY DOMAIN PERSISTENT HOMOLOGY. Ben Cassidy, Caroline Rae, Victor Solo
Thresholding (or selecting significance level) is problematic, particularly when the underlying system has a combination of small-scale features, some of which are noise artifacts, and some of which are critically important.  %YS bar code built with honology  help in that, short bars are noise, long ones are persistent features. 
% the approach of my eo ec paper:
One method for working around this difficulty is filtrations, which record the results of every possible binarization of the network (simplicial complexes building by doing incremental threhold or ball size), along with the associated threshold values. With this approach we can use first order measure of structure in networks (or simplicial complexes) and  “second order” measures as functions of edge weight. Such functions carry information, for example, in their rate of change, where sudden phase transitions in network structure as one varies the threshold can indicate the presence of modules or rich clubs in networks. The area under such curves can help us characterize the geometric structure in the activity of (hippocampal) neural populations, without appealing to external stimuli or receptive field ( \cite{giusti2015clique})

Dynamic network data are easily available from multiple disciplines, not only the life science (genomic) but specially from the social science, for example finances, social networks. A new approach for network comparison is persistent homology. Persistent homology is the analysis of the connectivity pattern as the threshold varies. Thus, it avoids the ad-hoc specification of a threshold for the study of the overlapping network patterns and the rates at which they vary.

An alternative approach is persistent homology consists on studying the architecture as it evolves with every new threshold.
A pair of nodes a, b are linked if $d(y_a, y_b) \leq \delta$. When $\delta = 0$ the network is totally disconnected and has N (number of nodes) components, the number of connected components increases as $\delta$ increases until the threshold is $\delta_{max}$ in which case the network is fully connected. It is the continuous variation of $\delta$ what gives us a graph filtration. The easiest way to show this graphically is to plot delta in one axis and the number of connected components in the other axis (barcode). (we can also use dendograms, where $\delta$ is the cut off). 

The Rips complex is the main algebraic representation of the entire brain and the multiscale networks are the Rips filtration which is a sequence of nested Rips complexes over different scales. The Betti numbers e.g. connected components, holes and voids, capture the topological information. 

The \textit{barcode} is the visualization of the changes in the Betti number over the Rips filtration. Thus, the barcode is the topological invariance representation of the network change over filtration, so it lacks geometric information about the nodes, in order to have that we can incorporate the node information to the barcode to obtain a dendogram. Then we can compare the distance between network by calculating the Gromov distance between dendograms.

The main importance of this approach is that it allows to have a multiscale network framework avoiding the threshold determination problem.
This graph filtration approach along with other persistent homology framework can be used to differentiate between classes of networks, so it can be a sort of hierarchical classifier. For example, used in differentiating between glucose metabolic network using 102 ROI from FDG-PET from ADHD, ASD and pediatric controls.

We have a measurement set denoted as $X= \{ x_1 ... x_p\}$ where p is the number of ROIs. The measurement $x_i$ can be assumed to be normally distributed $x_i ~ N(0,1)$, the distance between two measurements can be, for example,calculated with the Pearson correlation $c_{X}(x_i,x_j= 1 - corr(x_i,x_j))$. 
%YS cor should be in abs value
The measurement set X and the distance $c_X$ form the metric space $(X,c_X)$. The network constructed by thresholding correlation between nodes is as follows:
there is a link between two nodes if $c(x_i,x_j) \leq \delta$, for some threshold $\delta$. The collection of the edges and the nodes gives us a binary network $B(X,\delta)$ for threshold $\delta$ which is the same as the network $G(V,E)$. Note that here the threshold is variable not fixed as in the orthodox approach based on graph theory a la Sporns.

The basic concepts of persistent homology in relation to network are
\begin{itemize}
\item network as a simplicial complex
\item multiscale network as graph filtration and dendogram 
\end{itemize}

For p ROIs there are $2^p$ possible subsets that can be a possible topology, the power set is thus a representation of the underlying topology.
Importantly, every metric space is a topological space, then the metric space $B(X, \delta)$ is a topological space. 
Given a (measurement) set X and a rule of connections, the topological space is a simplicial complex and its element is a simplex. %which element
The important point is that a binary network is a simplicial complex consisting on 0-simplices (nodes) and 1-simplices (edges) and so on.

A Rips complex is a kind of simplicial complex. Given a set X, the Rips complex $R(X,\delta)$ is a simplicial complex whose k-simplices correspond to k+1-tuples of points that are pairwise within distance $\delta$. 
Now, while the binary networks have at most 1-semplices the Rips complex has at most p-1-simplices, so the Rips complex can have faces as well. Trivially, when we use the same distance to build the metric, we have $B(X,\delta) \in R(X,\delta)$

A node is a 0-simplex an edge a 1-simplex a triangle a 2-simplex and a complete graph with p nodes represent the edges of a p-1 simplex.

We have studied so far (i) "network as simplicial complex" for a fixed threshold, now we can study the relationship between networks as the threshold changes. Thus, for  binary network $B(X,\delta)$ we obtain the sequence of networks $B(X,\delta_0), ...,B(X,\delta_n)$. For the Rips complex as $\delta$ increases the Rips complex becomes larger, that is, $R(X,\delta_0 \in R(X,\delta_1) ...)$ for $\delta_0 \leq \delta_1 ...$. As the filtration value $\delta$ changes the topological properties of the Rips complex changes. Connected components are merged during Rips filtration is identical to the clustering in a dendogram.

Since metric space is a topological space we can compare two networks or topological spaces with topological measures rather than graph theoretic measures like assortativity, betweenness centrality etc.
The Gramov-Hausdorff distance allows us to measure the distance between two metric spaces, it just finds the edge that is most different in the two networks. For example, for the node sets X and Y since both are projected to the same template, then the node $x_i \in X$ is simply map to node $y_i \in Y$. Therefore the GH distance can be trivially discretized as 
\begin{equation}
d_{}(X,Y) = \frac{1}{2}_{x_i \in X, y_j \in Y} max |d_X(x_i,x_j) - d_Y(y_i,y_j))|
\end{equation} 
% Gromov distance in python http://stackoverflow.com/questions/30706079/hausdorff-distance-between-3d-grids
Since metric space is a topological space we can compare two networks or topological spaces with topological measures rather than graph theoretic measures like assortativity, betweenness centrality etc.
The Gramov-Hausdorff distance allows us to measure the distance between two metric spaces, it just finds the edge that is most different in the two networks. For example, for the node sets X and Y since both are projected to the same template, then the node $x_i \in X$ is simply map to node $y_i \in Y$. Therefore the GH distance can be trivially discretized as 
\begin{equation}
d_{}(X,Y) = \frac{1}{2}_{x_i \in X, y_j \in Y} max |d_X(x_i,x_j) - d_Y(y_i,y_j))|
\end{equation} 
% Gromov distance in python http://stackoverflow.com/questions/30706079/hausdorff-distance-between-3d-grids

\section{Limitations of the geometric network approach}
\label{se:s2}
We refer to Geometric Network approach to the design of networks with an unique threshold of choice. This is contrast with the algebraic topological approach which as we will show is better equipped to model complex biological networks. 

\subsection{The curse of dimensionality}
The curse of dimensionality refers to the situation in which the larger is the dataset the more sparse the data become. The rationale for this phenomenon is simple. For a 1-dimensional space, let us say that I need n random points to guarantee that they are separated a distance d. For a 2-dimensional space, I will need $n^2$ random points to guarantee a separation d, and for a 3 dimensional space $n^3$ and so on. Thus, the amount of data needed grows exponentially with the dimensionality. Technically speaking, this can be rephrased as local smoothing methods tend to perform poorly for multivariate data \cite{stone1982optimal},\cite{lavergne2008breaking}.  
%YS put a chart?

\subsection{Clustering}

Cluster analysis or clustering is the mapping of a measurement set into subsets called clusters, so that observations in the same cluster are similar (points within each cluster are similar to each other and points from different clusters are dissimilar).
The notion of similarity can be defined quite liberally with the selection of a distance between any two data points. Thus, a measurement set $X$ and a distance $d$ form a metric space $(X,d)$. 
A distance $d$ between two vectors $x,y$ in order to be such (metric) needs to fulfill the requirements: $d(x,y)$ is real non-negative and finite, is symmetric $d(x,y)=d(y,x)$, is unique $d(x,y)=0$ only if $x=y$ and holds the triangle inequality $d(x,z) \leq  d(x,y) + d(y,z)$.
There are many algorithms that construct clusterings on the metric space, for example k-means, fuzzy C-means, hierarchical clustering, spectral clustering etc. 

Although cluster analysis has become a principle technique of knowledge discovery in a wide range of fields, including social networks, microarray gene expression \cite{nugent2010overview} and classification of brain disorders \cite{arbabshirani2017single} to cite a few, its implementation is not free of pitfalls and ambiguities \cite{carlsson2009topology}, \cite{ronan2016avoiding}.
For example, the arbitrariness in the choice of the threshold result in the lack of robustness in the data partition. Indeed, little is known
about consistency of most clustering algorithms \cite{von2008consistency}.
% functoriality

A clustering function f takes a metric space $(\mathcal{X},d)$ and returns a partition of the underlying set $\mathcal{X}$, $f:(\mathcal{X},d) \to \prod(\mathcal{X},d)$. Kleinberg's impossibility theorem for clustering demonstrates that there is no clustering function satisfying three simple properties: scale-invariance, richness and consistency \footnote{The Arrow's impossibility theorem is another example of an axiomatic approach,in this case in the social science. Arrow's impossibility theorem states that 
 no rank-order voting system can be designed that always satisfies three fairness criteria \cite{arrow1950difficulty}}:
 % 
\begin{enumerate}
\item[i] Scale-invariance: An ideal clustering function does not change its result when the data are scaled. For example, by changing the metric distance from miles to kilometers the clustering function $f$ produces same result of for any scale $ f(d) = f(\alpha d)$.
\item[ii] Richness: Any partition of a finite set of points $\mathcal{X}$, $\prod(\mathcal{X},d$ can be created for some metric $d$ on $\mathcal{X}$ 
\item[iii] Consistency:  If we stretch the data so that the distances between clusters increases and/or the distances within clusters decreases, then  the clustering function produces the same result. Formally, let $d$ and $d'$ be two distance function, then if for any pair of points $(i,j)$ in the same cluster,$d(i,f) \geq d'(i,j)$ and for every pair in different clusters $d(i,f) \leq d'(i,j)$ then clustering does not change, $f(d)= f(d')$
\end{enumerate}
% YS : Draw chart
%http://alexhwilliams.info/itsneuronalblog/2015/10/01/clustering2/ 
What this important result teaches us is that no clustering function that satisfies the axioms of scale-invariance, richness and consistency.
Why this theorem is so poorly known, we do not know. It certainly shed light in the limitation and drawbacks of clustering algorithms. 
For example, in k-means, one of the most important clustering methods, the number of clusters is fixed to k, that is, there is no a "natural" number of clusters to be discovered, it is rather an assumption externally imposed by the modeler. Having the Kleinberg's theorem in mind, we can see that the question of how many clusters do the data set needs is ill-posed. It is similar to asking how many scales exist in the data. The answer is that unless there is  scale-invariance, structure emerges at different scales. 

\subsection{Dyadic relationship}
A graph consists of a set of vertices and a specified collection of pairs of vertices, called edges. An edge between two vertices exist when the distance between the relevant properties of those vertices within a certain threshold. Thus, a graph is a system of dyadic relationships and, very importantly, is unable to encode higher order interactions. For example, three nodes in a graph can work as a pacemaker or be simultaneously active and there is nothing in the graph-theoretic formalism that will tell the difference.
% put Chart

One could argue that this is not exactly true if we define the distance using coherency.
% formula of coherency
Coherency will identify the phase-lag and will distinguish between the pacemarker model (time-lag) and the simultaneous (zero-lag). %YS
Coherency analysis is capable of characterize coactivity, but we still have a problem when   when we want to express the relationship in networks. Because a network can only describe dyadic relationships between population elements, any binary coactivity network constructed from such observations would necessarily be identical. 
It does not matter which distance is used, correlation, coherency etc. at the end the network formalism will force us to build dyadic relationships.

The crux of the matter is to have available a mathematical formalism able that capture higher order interactions. More versatile versatile language able to explicitly encoding the multiple (other than double, triple etc.) coactivity pattern (Fig 1e) exist. 
The next section will introduce the simplicial complex, a model capable of encoding higher order interactions.% YS kind of literal, reformular
 
\section{}
\label{se:s3}
%Alegebraic topology




\section{Applications}
\label{se:s4}

\subsubsection*{Time series}   
The computation of distances between nodes when they represent time series needs special consideration. Time-series networks where each node has a record of scalar time series is pervasive in brain functioning modeling, for example electrophysiology (EEG) and functional magnetic resonance imaging (fMRI).
Metric distance between nodes is unsuited for networks of autocorrelated time-series. 
% YS: why?because they are autocorrelated?

The most basic distance between two nodes or recorded time series is $d(x,y) = \sqrt{1 - \rho_{x,y}}$, where $\rho_{x,y}=\frac{cov(x,y)}{var(x)var(y)}$.  From a signal processing point of view, we can calculate the distance as the closeness or how well can we predict one with the other, eg. predicting $y_t$ from $x_t$ (regression problem). The prediction error is $e_{y|x} = y_t - \alpha x_t$, where alpha $\frac{cov(x,y)}{var(x)}$.
The next step is to capture autocorrelation using pointwise frequency coherence measures.

For Cassidy, Rae and Solo there are two type of distances, marginal and joint distances. Marginal distances, for example, the Mahalanobis distance is based on marginal distribution statistics such as means, covariances and densities. For example, a marginal distance between two nodes of recorded time series $(x,y)$ can be calculated as, $d(x,y) = \sqrt{1 - \rho_{x,y}}$, where $\rho_{x,y}=\frac{cov(x,y)}{var(x)var(y)}$. 
From a signal processing point of view, we can calculate the distance as the closeness or how well can we predict one with the other, e.g. predicting $y$ from $x$ (regression problem). The prediction error is $e_{y|x} = y - \alpha x$, where alpha $\frac{cov(x,y)}{var(x)}$. For network comparison, marginal distances are inappropriate because the data are collected jointly. To understand this point we need to have present that marginal probability may be thought of as an unconditional probability. 
Marginal distance are unaffected by the threshold choice we can't explain away the other variables as marginal probability assumes.
%YS: put example

Joint distances, on the other hand, are based on joint statistics like cross spectra, mutual information and correlation. Pearl makes an important distinction between statistical and causal notions. Regression, Controlling for (conditioning), odd and risk ratios, Granger causality (Collapsability), Propensity score (Rubin's propensity score matching, make the groups receiving treatment and not-treatment more comparable) are all statistical notions based on joint distributions. 
Causal notions are spurious correlations (volume conduction), randomization, confounding factors, explanatory variables.


Topological neuroscience is an untapped resource for empirical neuroscientists.
Roughly speaking, the uses of topology in neuroscience can be categorized into three (overlapping) themes: (i) traditional topological data analysis applied to neuroscience; (ii) an upgrade to network science; and (iii) understanding the neural code \cite{curto2016can}.

\begin{enumerate}
\item  Calculate the statistics of the shape of point cloud data and more importantly the persistent homology (simplicial complex captures the information of intersecting balls, the simplicial complex can be Cech complex and Rips complex).
\item  "upgrade" of the network model: instead of a graph, one considers a simplicial complex. The higher-order simplices correspond to cliques (all-to-all connected subgraphs)
\item Suppose we are monitoring with painstaking detail the electrical firing at every time point of a neuron, what governs the behavior of this cell? the network, bien sure!  Now, imagine that we could monitor the activity of all the other neurons, and we knew exactly the pattern of connections between them, and were blessed with an excellent model describing all relevant dynamics, then (maybe?) we would be able to predict when our neuron will fire. This seems a far away project, but monitoring  single cells as we do now seemed impossible in the 50's. %lit
\end{enumerate}

Hubel and Wiesel pioneering work in the visual cortex inserted a microelectrode in the primary visual cortex of an anesthesized cat, but they could not monitor, let alone control the activity of neighboring cells, they could just listen to one neuron at a time, what they could control was the stimulus (bars with certain angle). They found orientation-tuned neurons.

O'Keefe made a remarkable discovery on the same lines, but more striking, because in the hippocampus there is not an obvious sensory path, but he reported that hippocampal place cells responded selectively to spatial locations.
With this discovery and with the Grid cells later on mind, we may ask are neurons vertices in a network or autonomous device (sensors) attuned with the outside world? Both models can be valid but yield different conceptual and methodological consequences. The first view, neurons are nodes in a network, is endorsed by neural networks and the last by neural coding. In the neural network paradigm the neuron behavior is described via network properties and in neural coding theory the neuronal behavior is a black box with the focus in the relationship between the external stimulus and the neural activity.

We know about the neural coding of neurons but not much abut the neural coding of networks, since it seems that they belong to two different conceptual schemes. But topology, a mathematical theory in which neuroscientists are not trained at all, may be the natural tool for understanding the neural code of brain networks.

For place cells, when simultaneous recordings of place cells became possible, it was shown via statistical inference (using previously measured place fields) that the animal's position could indeed be inferred from population place cell activity. The place cell code, thus, naturally reflects the topology of the represented space.

% Curto review for neural coding (place cells)
\cite{curto2016can} Some have argued that hippocampal place cell code is fundamentally topological in nature [12, 6] while others have argued that considerable geometric information is also present and can be extracted using topological methods [9, 18]. Dabaghian have worked towards disambiguate geometric and topological information: the place fields respected topological aspects of the environment more than metric features \cite{dabaghian2014reconceiving}. Also, about the grid cells, it turns out that the space represented by grid cells is not the full environment, but a torus.

\cite{giusti2016two} The use of networks is predicated on a critical simplifying assumption: that the quintessential unit of interest in the brain is a dyad – two nodes (neurons or brain regions) connected by an edge. While rarely mentioned, this fundamental assumption inherently
limits the types of neural structure and function that graphs can be used to model. Simplicial complexes (algebraic topology) is a generalization of graphs that overcomes these limitations.

The wealth of studies linking network models with cognition, disease \cite{stefan2013epileptic}, \cite{stam2014modern} and novel biomarkers points out to a paradigm shift in systems, cognitive, and clinical neuroscience: namely, that brain function and alteration are inherently networked phenomena.
% Dialectics, so far focus on one side, the bivariate, next step is the poly variate, when we master that, interventional attack will make sense (causal)
% the road towards causation pass by mastering the polyadic structure of the system
% Dyadic/Polyadic, bivariate/multivariate.

% In large-scale neuroimaging, cognitive functions appear to be performed by a distributed set of brain regions and even at a smaller scale, the spatiotemporal patterns of interactions between a few neurons is thought to underlie basic information coding (Szatmary and Izhikevich 2010).%no entiendo izi paper

Using the simple observation that place fields corresponding to nearby locations will overlap, the authors conclude that neurons corresponding to those fields will tend to be co-active. Importantly, because of their inherently non-pairwise nature, coactivation patterns of neurons or brain regions can be naturally encoded as simplicial complexes. 
Using the aptly (but coincidentally) named “Nerve Theorem” from algebraic topology, one can work backward from observed coactivity patterns to recover the intersection pattern of the receptive fields, describing a topological map of the animal’s environment.
The topological approach serve to decode maps of the environment from observed cell activity.

Dabaghian \cite{dabaghian2014reconceiving} shows that hippocampal place cells may ultimately be interested in a space's topological qualities (its connectivity) more than its geometry (distances and angles). After deforming an U track, the resulting place fields preserved the relative sequence of places visited along the track but did not vary with the metrical features of the track or the direction of the rat's movement.
Dabaghian claims that the paper provides evidence that hippocampal networks represent spatial contiguities and not 2D Cartesian maps created by path integration. The experiment directly pitted against one another these two hypotheses by having rats run in a variable shaped track in the near-dark, and CA1 place cells tracked the linear contiguities much better than the 2D shapes created by various configurations. 

In (Ellis and Klein 2014), the authors study the frequency of observation of coactivity patterns in fMRI recordings to extract fundamental computational units. Even when those regions which are coactive will change dynamically over time, cohesive functional units will appear more often than those that happen coincidental, though it is impossible to set an a priori threshold for the significance of such observations.
% we dont solve here the threshold, selection significance problem, it is in reality a problem of scale?
Real geometry is that that persist against changes of scale to understand the shape we need to dynamically change the threshold.
%https://www.youtube.com/watch?v=kctyag2Xi8o


Dotko and Markram use methods from algebraic topology that can deal more effectively with he immense complexity of biological networks 
as exemplified in the reconstruction and simulation of neocortical microcircuitry, comprising 8 million connections between 31,000 neurons \cite{markram2015reconstruction}. Dotko and Markram is the first algebraic topological analysis of structural connectomics and connectomics-based spatio-temporal activity in a biologically realistic neural microcircuit.  

\section{Discussion}
%YS Ojo literal
Geometric questions have been pondered since the Greeks or even before, however, topological ones are only one century old  \cite{edelsbrunner2010computational}.  A geometric space is defined via geometric measures (distances, areas, angles, etc.), and the relationships that are built on them. A topological space, on the other hand, defines a set of qualitative relationships between its parts: adjacency, containment, connectedness, overlap, cover, etc. \cite{aleksandrov1999mathematics}. Thus, geometric space is defined in terms of quantitative features and the topological space uses  
in terms of qualitative relationships.  Both set of features, the metrical and the topological are related but the relation is asymetrical, that is, topological (qualitative) features can be deduced from metrical (quantitative) ones but the inverse is not true. For example, knowing the location and length of two segments we know if they overlap, but knowing that one segment is adjacent to another one does not say nothing about where are they located or their length. 

% Krakauer
%With the advent of high-throughput genomics, transcriptomics, proteomics, and metabolomics, and functional imaging we have witnessed a technological revolution in biology have gone hand by hand with the rise of bioinformatics and ML prediciton.
%But there has been a lack of complementary conceptual theory that could help us organize the flood of facts. An emphasis on models, rather than theory.
%A persistent problem in biology is (contrary to statistichal mechanics) that regularities exist at aggregate levels of description, therefore an qualitatively different theory is required to explain these emergent phenomena than the theory describing the underlying microscopic dynamics (Anderson, 1972).
%In biology, unlike for traditional physical and chemical phenomena, many of the spatial and temporal scales interact. In physics, nuclear forces can be neglected when calculating plane- tary orbits as these are screened off over large distances. In biology, however, the lowest levels can have a direct impact on the highest levels (and vice versa), as in the case of genes that influence behavior and social structures and behavior that influence gene expression patterns.
%PROBABILITY, RANDOM VARIABLES, AND STOCHASTIC PROCESSES Papoulis, Athanasios Pillai, Unnikrishna
There is no conflict between causality and randomness or between determinism and probability if we agree, as we must, that scientific theories are not discoveries of the laws of nature but rather inventions of the human mind. Their consequences are presented in deterministic form if we examine the results of a single trial; they are presented as probabilistic statements if we are interested in averages of many trials. Thus, there is always uncertainty, also in the so called deterministic setting where the uncertainties are "with certain errors and in certain ranges of the relevant parameters" and in the stochastic "with a high degree of certainty if the number of trials is large enough." Classical mechanics holds within an error e provided that the neglected factors are smaller than $\varphi$.
%A biological relativity view of the relationships between genomes and phenotypes. (D. Noble)
There is no privileged scale of causality in biology to clarify the relationships between genomes and phenotypes.  The idea that genetic causes are primary views the genome as a program. Initially, that view was vindicated by the discovery of mutations and knockouts that have large and specific effects on the phenotype. But we now know that these form the minority of cases. Many changes at the genome level are buffered by robust networks of interactions in cells, tissues and organs.  The 'differential' view of genetics therefore fails because it is too restrictive. An 'integral' view, using reverse engineering from systems biological models to quantify contributions to function, can solve this problem. % downward causation

\subsection{Causality}
GCM (J.Pearl) Causality is inferences under change. A causal model is a symbolic system that encodes the invariance, what remains constant.
Conditioning probabilities do not work for intervention, if they did, curing symptoms would cure diseases
%https://youtu.be/RPgvfSeQB8A 
Dichotomy between causal and statistical notions.

Statistical notions or notions that can be defined with a joint distribution are Regression, Controlling for (conditioning), odd and risk ratios, Granger causality (Collapsability), Propensity score (Rubin's propensity score matching, make the groups receiving treatment and not-treatment more comparable).
Causal notions are spurious correlations (volume conduction pb), randomnization (you cant tell from the observation, we need additional assumptions, we need an intervention), confounding, explanatory variables.
A confound, a lurking variable or a confounder is a variable in a statistical model that correlates (directly or inversely) with both the dependent variable and an independent variable,  in a way that "explains away" some or all of the correlation between these two variables (if z affects to x and y, and x and y are not correlated it may look that x and y are correlated (z has explained away xy , confounding). note that confounding is a causal notion while correlation is not.

%http://voxeu.org/article/limitations-randomised-controlled-trials
Cartwhright, no causes in no causes out, in order to have causal conclusions you need to have causal assumptions. Example of statistical assumptions, normal distribution, independence ... but we need the judgemental causal assumptions because only those encode the invariance.

The important point is that causal assumptions cannot be expressed in the language of standard statistics. In 1920 Sewell Wright (and the early econometricians) created the SEM.
%http://ftp.cs.ucla.edu/pub/stat_ser/r370.pdf  The Causal Foundations of Structural Equation Modeling
The paper starts with "The role of causality in SEM research is widely perceived to be, on the one hand, of pivotal methodological importance and, on the other hand, confusing, enigmatic, and controversial.", and the same can be said about causality. The origins are more ilustrious, Aristotle, Hume etc but the contrast between the insidious and necessary and the enigmatic is there. 

Causal effects in observational studies can only be substantiated from a combination of data and untested theoretical assumptions, not from the data alone.
%Many SEM textbooks have subsequently considered the term “causal modeling” to be an outdated misnomer (e.g., Kelloway, 1998, p. 8), giving clear preference to causality-free nomenclature such as “covariance structure,” “regression analysis,” or “si- multaneous equations.” 
%Comparing structural equation models to the potential-outcome framework, Sobel (2008) asserts that “in general (even in randomized studies), the structural and causal parameters are not equal, implying that the structural parameters should not be interpreted as effect.”
Rubin and Sobel are from the potential-outcome framework. also Paul Holland which states that the equation $y = a +bx+ \epsilon$ what really means is a way to expresses the $p(y|x)$. But for Pearl, the  structural interpretation of this equation has nothing to do with the conditional distribution of ${y}$ given  ${x}$ ; rather, it conveys causal information that is orthogonal to the statistical properties of  ${x}$  and  ${y}$. For Pearl, the SEM language in its nonparametric form offers a mathematically equivalent alternative to the potential-outcome framework that Holland and Sobel advocate for causal inference. This explains why SEM retains its status as the prime language for causal and counterfactual analysis.

\cite{wiedermann2016statistics}
Not to confuse X(correlation) with Y(causation) falls a little flat, specially since we dont know what causation means, correlation on the other hand we know many, maybe too many ways to calculate ad plot it in Matlab,R, SPSS or any of the plethora of statistical modeling software available these days. 
The counterfactual or “interventionist” account lead by Woodward and Pearl is gaining momentum.

%Statistics and Causality: Methods for Applied Empirical Research. Chapter 11
In cell biology causality networks are also called gene regulatory networks. In neuroscience, causality networks are widely used to express the temporal interactions between various regions of the brain.
%Andrew Lo physics envy https://arxiv.org/pdf/1003.2688.pdf
\subsubsection{Approaches for the reconstruction of the causal relations}
Hypergraphs, can record any possible collection of relations but it is precisely this degree of generality leads to a combinatorial explosion in systems of modest size. 
Giusti et al, argues that simplicial complexes gives a compact and computable encoding of relations between arbitrarily large subgroups while retaining access to a host of quantitative tools for detecting and analyzing the structure of the systems they encode. In particular, the homology \footnote{Names of topological objects have a seemingly pathological tendency to conflict with terms in biology, so long have the two subjects been separated. Mathematical homology has no a priori relationship to the usual biological notion of homology} of a simplicial complex is a collection of topological features called cycles that one can extract from the complex.

In 1969 Granger introduced a method to quantify temporal-causal relations among time series measurements. He introduced Wiener’s concept of causality into the analysis of time series (Wiener,1956)
%Wiener, N. (1956) The theory of prediction, in Modern Mathematics for Engineers
Granger causality (LR model) is impaired by several crucial problems of discovering latent confounding effect, missing counterfactual reasoning and capturing instantaneous and nonlinear causal relationships (Spirtes et al., 2001, Pearl, 2009, Bahadori and Liu, 2013b).
Granger is not causal model rather it is a temporal dependence discovery method.
GC characterize the extent to which a process influences another process and builds upon the notion of incremental predictability. Thus $x_t$ Granger causes another process $y_t$ if the future values of y can be better predicted with past values of x and y than with past values of y alone.
%Bunge book Causality
The historical antecedent of the philosophy behind Grange's causality is the father of modern economics, Paul Samuelson [34] who adopted the regular-succession-in-time view of causation  and, as a consequence, regards difference equations and differential equations with time as the independent variable, as causal laws. 
However, there need to be nothing causal about a regular sequence of events and, unless the equations describing it are enriched with semantic assumptions pointing to causal factors, the equations are neither causal nor noncausal. 


The bivariate model is:
$y_t = a_0 \sum_{k=1}^{L}b_{1k}y_{t-k} \sum_{k=1}^{L}b_{2k}x_{t-k} + \phi_t$
where $\phi_t$ are uncorrelated random  variables of mean 0 and variance $\sigma^2$, L is the time lag and tells us the number of past values taken into consideration. The null hypothesis that $x_t$ does not Granger cause $y_t$, that is,$b_{2k} = 0$ for $k=1..L$.

It can be straightforwardly extended to p-dimensional multivariate time series $x_t \in R^{p \times l }$ (p number of variables, for example, space locations (roi) l is the length of the time series).
Hume cause must precede the effect.
The approximation problem
$x_t \sim \sum_{p=1}^{P} \sum_{l=1}^{L}\beta_{l}^{p} x_{t-l}^{p}$
The approximation problem can be approximated with least squares to estimate the coefficients $\beta_{l}^{p}$ from a system of linear equations.


Granger causality on gene regulatory networks with a large p may not lead to satisfactory results (Lozano et al., 2009). The statistical significant tests are inefficient, while they lead to higher chances of spurious correlations. Moreover, the high dimensionality of biological data leads to further challenges (sparse parameter vector). To address this issue, various variable selection procedures can be applied (Lasso, LARS, elastic nets).
Lasso is an alternative regularized version of least squares, which, in addition to the minimization of the residual sum of squares, imposes an norm on the coefficients.LARS (least angle regression) is a less greedy version of traditional forward selection method is computationally less intensive compared to Lasso makes it widely used in variable selection problems. However, for highly correlated variables, Lasso and LARS tend to select only one variable instead of the whole group. The the elastic net method addresses this problem outperformed Lasso in terms of prediction error for correlated data.

The problem of the reconstruction of a gene regulatory network belongs to the class of inverse problems with high-dimensional data set and sparse number of measurements. A general inverse problem can be seen as an operator equation $y = A \beta$ where y is the effect, $\beta$ is the cause and A is the model between the cause and its effect. The approximation problem can be seen as an inverse problem.
In practice, one has to take into account that the data y  are noisy. The data y deviate from the ideal data $y*$ and the norm $||y - y* ||$ is the noise. 
Inverse problems are often ill-posed, which that the above Equation (using the noisy data y) may have no solution, or more than one Regularization methods are proposed to deal with the ill-posedness of inverse problem (e.g. Tikhonov)

% from Bunge Causality, notes 3rd edition
As for final causation, laughed away by Rabelais, Bacon, and Spinoza, it is now back in biology under the name of ‘teleonomy’ (cf. Mayr [19], Monod [17], Ayala [20]). The simplest difference of bacteria and a nonliving thing is that the former is goal-directed, or has a ‘teleonomic project’ built into it. downward causation: term ‘causation’ is being misused in this context, for what is at stake is a multilevel system.
The truth seems to lie in a synthesis of the upward and the downward views, neither of these should be formulated in terms of causation because levels, being sets, cannot act upon one another.
This is precisely why we need a mathematical language other than sets (networks), for example, complexes, categories etc. because once you pick up your threshold you have fixed the  set, and sets cannot act upon each other.
%YS how can i translate "sets dont act upon eachother" in mathematical formalism?

% http://www.nature.com/nature/journal/v541/n7636/full/541156a.html Poldrack review of "Sex, Lies, and Brain Scans: How fMRI Reveals What Really Goes on in our Minds"
fmri reveals only correlations. Just because an area is active when one experiences fear, it doesn't mean that that region is necessarily involved in the experience of fear. Causal necessity can be demonstrated only by manipulating the function of a brain region, either through brain stimulation or the study of brain lesions. For instance, many studies show that the ventromedial prefrontal cortex (important in value-based decision-making) is activated when research participants are considering how much they are willing to pay for consumer goods. But recent work has found that some people with lesions to this area show no impairment in such abilities (A. R. Vaidya and L. K. Fellows Nature Commun. 6, 10120; 2015).
%future works
Signal processing in graphs: The emerging field of signal processing on graphs merges algebraic and spectral graph theoretic concepts with computational harmonic analysis to process such signals on graphs. 
\cite{shuman2013emerging}

\begin{acknowledgements}
If you'd like to thank anyone, place your comments here
and remove the percent signs.
\end{acknowledgements}

% BibTeX users please use
\bibliographystyle{spmpsci}
\bibliography{../bibliography-jgr/bibliojgr}   % name your BibTeX data base

\end{document}
% end of file template.tex

%%%%%%%%%%%%%%%%%%%%%%%%%%%%%%%%%%%%%%%%%%%%%%%%%%%%%%%%%%%%%%%%%%%%%%%%%%%%%%%%%%%%%%%%%%%%%%%%%%%%%%%%%%%%%%%%%%%%%%%%%%%%%%%%%%%%%%%%%%%%%%%%%%%%%%%%%%%%%%%%%%%%%%%%%

\section{Appendix}
Definitions:

%%%%% Lecture on Topology %https://www.youtube.com/watch?v=7G4SqIboeig Begin 
%at the coarsest level spaceset is a set, that is, it consists on a set of points,but this is not enough, something needs to ber added if we want to talk about for example continuity. We need to add structure on set to be able to work with continuity. The weakkes structure that can be establish in a set that allows a good definition of ciontinuity of maps, this minimal structure is topology. (motivation for using topology)
Definition -- Power set
The power set of a M is the set containing all subsets of M, including the empty set and M itself. For example, the power set of $M={1,2}$ is $ \mathcal{P}(M) = \{\emptyset, \{1\},\{2\},\{1,2\}\}$.
 
The axioms that define topological space can be given in terms of open sets or using the concept of neighborhood. Here we define topological space using open sets, for the neighborhood definition see \cite{brown2006topology}.

Definition -- homeomorphism
An homeomorphism is a continuous map that establishes a one-to-one correspondence between two objects. Two objects are homeomorphic if they can be deformed into each other by a continuous, invertible mapping.
A  map $f:M \to N$ is continuous iff the preimage of any open set is open.

The preimage or invserve image of a set V, $V \exists N$ defined in the map $f:M \to N$ is the set of elements in the domain M that land in V.
Example Draw figure 

Definition -- Continuous map:
To build a map we just need two sets, but to answer the question is the map continuous, we need topology. Whether a map, $f: M \to N$  is continuous depends on which topologies are chosen on both the domain set M and the target set N.

Definition: Let the topological spaces,  $(M,O_m)$ and $(N,O_N)$ and the map, $f:M \to N$ is called continuous with respect to the topology of $O_M$ and $O_N$ if for every set that is open in the target, $\forall \exists O_N$, the preimage is an open set in the domain, $preim_f(V) \exists O_M$ (the preimage(V) is the set of elements in the domain that land in V)

%min 52
Example: Check whether the map is continuous with respect tot he chosen topologies, we find the primages of the target
The composition of continuous maps is also continuous

Definition -- Convex Hull
Convex hull or convex envelope is the smallest convex set that contains all points in set X. Computing the convex hull means constructing an unambiguous, efficient representation of the required convex shape. The convex hull of the given points is identical to the set of all their convex combinations.
A convex combination is a linear combination of points (which can be vectors, scalars, or more generally points in an affine space) where all coefficients are non-negative and sum to 1. Formally give n points in real vector space, $x_1, ... x_n$ a convex combination is a point $y=a_{1}x_1 + ... + a_{n}x_n$ such that the sum of the weights is 1, $\sum a_i = 1$. We can also by the same token obtain the convex combination of probability distributions. 

For example, every convex combination of two points lies on the line segment between the points, then in an edge we have the convex hull or the 2-1(=1)-simplex. For three points, every convex combination will lie in the area, so the triangle is a convex hull or a 3-1(=2)-complex. %YS dont see this, why every linear combination must lie in the straight line or in the triangle for 3points?
% https://en.wikipedia.org/wiki/Convex_combination
%mean corrected (Scanner drift is corrected by detrending and mean correction is the simplest way to detrend data -normalizing to the mean signal intensity of each run other more complicated forms of detreanding is wavelets, polinomial interpolation etc.)



%Examples of simplicial complexes%%%%

%Poincaré argues that what science can attain is not the things themselves,
%as the naive dogmatists think, but only the relations between the things. Outside of these relations there is no knowable reality.
%Since the interaction with the world is described by information, it is by dealing with information that these systems most effectively persist.

%Researchers in big data applied to biomedicine are always trying to come to terms with the huge data flow available. However, the pressure for having better diagnosis and personalized treatment will be too strong to ignore. This customer's pressure will translate from clinicians to researchers, already is.
Examples of simplicial complexes applied to brain data are the clique complex, the concurrence complex \cite{ellis2012describing}; \cite{curto2008cell}; Dowker 1952), its Dowker dual (Dowker 1952), and the independence complex (Kozlov 2007).
Of course, a graph is a simplicial complex. Graphs are good for encoding dyadic relationships, Clique Complex can do polyadic canonical extension of graphs, Concurrence/Dual complex is good at capturing relationships between two variables of interest eg: time and activity or activity in two distant regions and  Independence 

Clique complex:  a clique is all-to-all connected subgraph. Given a graph,replace every vertice replaces every clique by a simplex on the vertices participating in the clique. This procedure produces a clique complex that has been used for neural analysis of hippocapal cells in both spatial and non spatial behavior \cite{giusti2015clique} and in general for correlation and coherence maps of for example fMRI data.
%This application demonstrates that simplicial complexes are sensitive to organizational principles that are hidden to graph statistics, and can be used to infer parsimonious rules for information encoding in neural systems. 
Thus, Clique complexes precisely encode the topological features present in a graph, so clique complexes "complete" graphs providing new views on the topological properties of graphs.
However, other types of simplicial complexes can be used to represent information that cannot be so encoded in a graph.

There are other complexes, like concurrent complexes (for coactivation but I dont see how they give more than coherence matrices) or Independence complex (hypergraph) to study system’s community structure (Bassett).
%Topological Data analysis (TDA) provides geometric summaries of our data. Data has shape and shape has meaning.
%Topology is about shape, and shape is the global realization of local constraints. %YS this is problematic because if topology is about shape why it is hard to get the shape from the topological properties. This motto cant referred to shape as geometric shape . all boils down to how you define the relationships, if a metric (euclidean) or some sort of similarity, in that last case one couldnt possibly reconstruct the shape

%https://youtu.be/gtFVdGb9Y8w?t=5m30s
%Given a metric on the rows, and the columns are the features and we can work our shape with the metric, but note that we can relax the study of shape and use similarity rather than metric.
%Once you have a metric + a threshold of statistical relevance the data has a (topological) shape, they occupy parts in the metric space, there is gaps, holes, flairs, bubbles...and with this we can stop and ask what the shape we just devised tells us about the data.
%Since traditional graph-theoretical methods may not be sufficient to understand the immense complexity of such a biological network, we explored whether methods from algebraic topology could provide a new perspective on its structural and functional organization. 
%An extension of this ideas is directedness of information flow has been used to investigate the relationship between simulated structural and functional neural networks \cite{dotko2016topological}. % Markram



 
"Marginal" distance because they used to be found by summing values in a table along rows or columns, and writing the sum in the margins of the table. 
%http://www.statisticshowto.com/marginal-distribution/  https://en.wikipedia.org/wiki/Marginal_distribution
%time series distance http://stats.stackexchange.com/questions/185912/alternate-distance-metrics-for-two-time-series

Joint distances on the other hand, are based on joint statistics like cross spectra, mutual information and correlation. If the data are collected separately in time and space then there is no point in calculating joint distance, rather we need marginal distance. On the other hand, for network comparison, marginal distances are inappropriate because the data are collected jointly. 
% YS: wy not, i have women and men and i want to know how they like fish or meat.
% I can day in day one and site 1 plus day 2 site 2 and calculate the marignal, p(someone like meat) p(somelike fish).
% I think this refers exclusively for time series
% for network analysis we cant do just marginal , because there p(a) is unconditional to any other event, that is, we must assume that the nodes (time series recirded at different (brain) location are not coupled)

%A confound, a lurking variable or a confounder is a variable in a statistical model that correlates (directly or inversely) with both the dependent variable and an independent variable,  in a way that "explains away" some or all of the correlation between these two variables (if z affects to x and y, and x and y are not correlated it may look that x and y are correlated (z has explained away xy , confounding). note that confounding is a causal notion while correlation is not. 

%pearl knows that joint, neither marginal nor conditional are non causal concept, so he created a new language for causal networks, but always networks, always dyadic, the problem of coactivity will be there, also the pb of the threshold

